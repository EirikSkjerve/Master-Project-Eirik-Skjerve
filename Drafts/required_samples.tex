\documentclass[12 pt]{article}        	%sets the font to 12 pt and says this is an article (as opposed to book or other documents)
\usepackage{amsfonts, amssymb}					% packages to get the fonts, symbols used in most math
\usepackage{algorithm}
\usepackage{algpseudocode}

\usepackage{amsthm}
\usepackage{mathtools, nccmath}
\DeclarePairedDelimiter{\nint}\lfloor\rceil

\usepackage[integrals]{wasysym}
\usepackage{latexsym}
\DeclareRobustCommand{\VAN}[3]{#2}
\usepackage{amssymb} 
\usepackage{amsmath}
\usepackage{cite}
\usepackage[english]{babel}
\usepackage{parskip}
\usepackage{relsize}
\usepackage{todonotes}
\usepackage{booktabs}  % For professional table rules
\usepackage{array}
%\usepackage{setspace}               		% Together with \doublespacing below allows for doublespacing of the document

\oddsidemargin=-0.5cm                 	% These three commands create the margins required for class
\setlength{\textwidth}{6.5in}         	%

\renewcommand{\baselinestretch}{1.5} 
\addtolength{\voffset}{-20pt}        		%
\addtolength{\headsep}{25pt}

\newcommand{\PP}[2][]{\mathcal{P}_{#1}(\mat{#2})}
\newcommand{\mat}[1]{\mathbf{#1}}
\renewcommand{\vec}[1]{\mathbf{#1}}
\newcommand{\GLnR}{\mathcal{GL}_{n}(\mathbb{R})}
\newcommand{\normdist}[2]{\mathcal{N}(#1, #2^2)}
\newcommand{\dgdist}{\mathcal{D}_{2\bb{Z}+c, \sigma}}
\newcommand{\bb}[1]{\mathbb{#1}}
\newcommand{\dgd}{\mathcal{D}}
\newcommand{\dgdi}{\widehat{\mathcal{D}}}
\renewcommand{\dot}[2]{\langle \vec{#1}, \vec{#2} \rangle}
\newcommand{\mom}[2]{mom_{#1, \mathbf{#2}}(\mathbf{w})}

\DeclarePairedDelimiter{\round}\lfloor \rceil

\begin{document}

We have computed $\mu_4$, i.e. the 4th moment of Z to be 2.999..., and that the difference ($\mu_4 - 3$) is about $2.1 \cdot 10^{-14}$ for parameter 256.
Let us now compute how many samples one would need to successfully distinguish $mom_{4, \mat{C}}(\vec{u})$ when $\vec{u} \in \pm \mat{C}$ and when $\vec{u} \not\in\pm\mat{C}$.
$mom_{4, \mat{C}}(\vec{u}) = \bb{E}[(\langle \vec{u}, \mat{C}\vec{x} \rangle ^4] = 3 + (\mu_4 - 3) \mathlarger{\sum_{i=1}^{n} \langle \vec{u}, \vec{c}_i \rangle ^4}$.
If $\vec{u} = \vec{c}_i$, the entire expression is $3 + (\mu_4 - 3)\cdot 1 = \mu_4$ because $\langle \vec{c}_i, \vec{c}_i \rangle = 1$ and $\langle \vec{c}_i, \vec{c}_j \rangle = 0$ for $i \neq j$. 
Otherwise, the sum $(\mathlarger{\sum_{i=1}^{n} \langle \vec{u}, \vec{c}_i \rangle ^4}) < 1$.


Say one wants to distinguish $mom_{4, \mat{C}}(\vec{c}_i)$ and $mom_{4, \mat{C}}(\vec{u})$ for $\vec{u} \not\in\pm\mat{C}$.
Let $y = \langle \vec{u}, \mat{C} \vec{x} \rangle$ and assume $y$ follows a standard normal distribution with $\mu = 0$ and $\sigma^2 = 1$.
Then we approximate the variance $\sigma^2$ and standard deviation $\sigma$ for $\langle \vec{u}, \mat{C}\vec{x} \rangle ^4$ as $\bb{E}[y^8] - \bb{E}[y^4]^2$.
Then $\sigma^2 \approx 105 - 9 = 96$ and $\sigma = \sqrt{96} \approx 9.79796$.

Now, to estimate number of samples we use formula for confidence intervals and Central Limit Theorem. Let $s$ denote required number of samples. Then 
\[
    s \geq \mathlarger{(\frac{Z_{\alpha/2} \cdot \sigma}{\mathsf{err}})^2}
\]
Now, we want $\mathsf{err}$ to be smaller than $2.1 \cdot 10^{-14}$ for parameter 256.
If we set $\mathsf{err}$ to be $10^{-14}$, and we want to determine a difference with, say, 75 \% confidence, we get:

\[
    s \geq (\frac{0.68 \cdot 9.79796}{10^{-14}})^2 \approx 4.439 \cdot 10^{29}
\]
where $0.68$ is from standard normal tables for 75 \%.
This result is way beyond the limit of acceptable transcript size, which for parameter 256 is $2^{32} \approx 4.3 \cdot 10^{9}$ and for non-challenge parameters 512 and 1024 is $2^{64} \approx 2 \cdot 10^{19}$.
One can therefore conclude that the attack will not be successful.

\end{document}

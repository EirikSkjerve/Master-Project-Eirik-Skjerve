\section{Hawk}
In this section we introduce the digital signature scheme Hawk \cite{HawkSpec24}.
As mentioned in the introduction, Hawk is a lattice based signature scheme that shares some key points with the digital signature scheme NTRU-sign \cite{HHPSW03}, the target for the original HPP attack.
The first part of this chapter, section 3.2, will give an introduction on how the Hawk scheme works on a high level. After this, the implementation of Hawk will be described in section 3.3, where more detail about 
specific subroutines and parts of the procedures will be given.

In these sections, some parts of Hawk as described in \cite{HawkSpec24} are left out due to the focus of this thesis. For example, Hawk utilizes compression of keys and signatures, 
% (which influence the key generation and signature generation procedures),
and some of their security considerations regarding a practical implementation are based on side channel attacks, both of which is not a concern in this thesis.
Moreover, their security analysis is based on experimental lattice reduction and security proofs of the underlying hard problems (namely the \textit{Lattice Isomorphism Problem} and 
the \textit{One More Shortest Vector Problem}), which, again, is not the focus of this thesis. 

% Perhaps the most notable similarity is the algebraic structure of private key $\mat{B}$. Recall that in NTRU, polynomials are defined 
% over $\bb{Z}[X] / (X^n - 1)$, whereas in Hawk they are defined over $\bb{Z}[X] / (X^n + 1)$.
% The private key in NTRU is defined as 
% \[
%     \begin{bmatrix}
%         f_0 & f_1 & \cdots & f_{n-1} & g_0 & g_1 & \cdots & g_{n-1} \\
%         f_{n-1} & f_{0} & \cdots & \cdots & g_{n-1} & f_{0} & \cdots \\
%         \cdots \\ 
%         f_{1} & f_{2} & \cdots & f_{n-1} & f_0 & g_{1} & g_{2} & \cdots & g_{n-1} & g_0 \\
%         F_0 & F_1 & \cdots & F_{n-1} & G_0 & G_1 & \cdots & G_{n-1} \\
%         F_{n-1} & F_{0} & \cdots & \cdots & G_{n-1} & F_{0} & \cdots \\
%         \cdots \\ 
%         F_{1} & F_{2} & \cdots & F_{n-1} & F_0 & G_{1} & G_{2} & \cdots & G_{n-1} & G_0 \\
%     \end{bmatrix}
% \]
\subsection{Overview}
In Hawk, we work with polynomials defined over the cyclotomic number field $\mathcal{K}_n = \bb{Q}[X]/(X^n +1)$ and its corresponding ring of integers
$\bb{Z}[X] / (X^n + 1)$. Such a polynomial $f \in \mathcal{K}_n$ can also be represented by a (column) vector of its coefficients in $\bb{Q}^n$ (in practical setting, coefficients are in $\bb{Z}$ instead of $\bb{Q}$), 
denoted $\mathsf{vec}(f)$, where the index of each coefficient corresponds to its power of $X$ in $f$. 
\footnote{This is a polynomial representation that is frequent in programming implementations when a dedicated algebra library is not used.}
Explicitly, if $f = a_0 X^0 + a_1 X^1 + \cdots + a_{n-1} X^{n-1}$
\[
    \mathsf{vec}(f) = 
    \begin{bmatrix}
        a_0 \\
        a_1 \\
        \cdots \\
        a_{n-1} \\
    \end{bmatrix}
\] 
We define a mapping 
\[\mathsf{rot:} \mathcal{K}_n \rightarrow \bb{Q}^{n \times n}, \mathsf{rot}(f) = 
\begin{bmatrix} 
    \mathsf{vec}(f) & \mathsf{vec}(fX) & \mathsf{vec}(fX^2) & \cdots & \mathsf{vec}(fX^{n-1})
\end{bmatrix}    
\]
\todo{This should maybe be in the background section...}
which enables one to construct a matrix in $\bb{Q}^{n \times n}$ from a single polynomial in $\mathcal{K}_n$. A vector $\mathsf{vec}(fX^k)$ to any power $k$ will 
henceforth be referred to as a "negacyclic shift" of $f$, since all entries with powers $X^{n-t}$ for $n-t + k  < n-1$ are shifted $k$ times, 
and entries with powers $X^{n-s}$ where $n-s + k \geq n$ will "go around" and be negated as $- X^{n-s +k \mod n}$.

The hash function used in Hawk is SHAKE256, an extendable output function (XOF) which enables arbitrary output length, unlike hash functions like SHA256 which 
has a static output size of 256 bits. This property is useful since SHAKE256 can be used in all three degrees of Hawk.

\subsection{Parameters}
Below is the parameters for Hawk degree 256, 512, and 1024. Note that this list only contains the relevant parameters for this thesis, and that many parameters relevant to 
compression and decompression are omitted.

\begin{table}[h]
    \centering
    \caption{Parameter sets for \textbf{Hawk}}
    \label{tab:hawk-parameters}
    \begin{tabular}{lccc}
        \toprule
        \textbf{Name} & \textbf{HAWK-256} & \textbf{HAWK-512} & \textbf{HAWK-1024} \\
        \midrule
        \multicolumn{1}{l}{Targeted security} & Challenge & NIST-I & NIST-V \\
        Bit security $\lambda$ & 64 & 128 & 256 \\
        % \midrule
        % Private key size $\text{privlen}_{\text{bits}}/8$ (bytes) & 96 & 184 & 360 \\
        % Public key size $\text{publen}_{\text{bits}}/8$ (bytes) & 450 & 1024 & 2440 \\
        % Signature size $\text{siglen}_{\text{bits}}/8$ (bytes) & 249 & 555 & 1221 \\
        \midrule
        Degree $n$ & 256 & 512 & 1024 \\ 
        Transcript size limit & $2^{32}$ & $2^{64}$ & $2^{64}$ \\ 
        Centered binomial $\eta$ for sampling ($f, g$) & 2 & 4 & 8 \\
        Signature std. dev. $\sigma_{\text{sign}}$ & $1.010$ & $1.278$ & $1.299$ \\
        Verification std. dev. $\sigma_{\text{verify}}$ & $1.042$ & $1.425$ & $1.571$ \\
        \bottomrule
    \end{tabular}
\end{table}
\todo{Insert more relevant parameters here}

\subsection{Hawk key pairs and key pair generation}
A Hawk private key is represented by the matrix
\[ \mat{B} = 
    \begin{bmatrix} 
        f & F \\
        g & G
    \end{bmatrix}
    \in \mathcal{K}_{n}^{2 \times 2}
\]
Here, entries are chosen such that $f$ and $g$ have small coefficients, and the equation $fG - Fg = 1 \mod X^{n} + 1$ holds, and consequently $\det(\mat{B}) = 1$, making $\mat{B}$ invertible over $\mathcal{K}_{n}^{2 \times 2}$.
The inverse of $\mat{B}$ is 
\[
    \mat{B}^{-1} =
    \begin{bmatrix}
        G & -F \\
        -g & f
    \end{bmatrix}
    \in \mathcal{K}_{n}^{2 \times 2}
\]
A Hawk public key is defined as 
\[
    \mat{Q} = 
    \begin{bmatrix}
        q_{00} & q_{01} \\
        q_{10} & q_{11}
    \end{bmatrix}
    =
    \mat{B}^* \mat{B}
    =
    \begin{bmatrix}
        f^* f + g^*g & f^* F + g^*G \\
        F^* f + G^*g & F^*F + G^*G
    \end{bmatrix}
\]
where $f^*$ denotes the Hermitian adjoint of $f$, explicitly $f^* = f_0 - f_{n-1}X - \cdots - f_{1}X^{n-1}$,
and $\mat{B}^*$ is the transpose of $\mat{B}$, $\mat{B}^T$ where each entry is taken the Hermitian adjoint of. Explicitly:
\[
    \mat{B}^* = 
    \begin{bmatrix}
        f^* & g^* \\
        F^* & G^*
    \end{bmatrix}
\]
We can also define $\mathsf{rot}$ on a matrix as the mapping 
\[\mathsf{rot}: \mathcal{K}_n^{2 \times 2} \rightarrow \bb{Q}^{2n \times 2n}, \ 
    \mathsf{rot}(
    \begin{bmatrix}
        f & F \\
        g & G
    \end{bmatrix}
    )
    =
    \begin{bmatrix}
        \mathsf{rot}(f) & \mathsf{rot}(F) \\
        \mathsf{rot}(g) & \mathsf{rot}(G)
    \end{bmatrix}
\]

Note that when working with $\mathsf{rot}(\mat{B})$ instead of $\mat{B}$, the Hermitian adjoint $\mat{B}^*$ corresponds to the transpose of $\mathsf{rot}(\mat{B})$, as 
$\mathsf{rot}(\mat{B}^*) = \mathsf{rot}(\mat{B})^T$. This is easy to show by observing that the first row of $\mathsf{rot}(\mat{B}^T)$ is equal to the first column of $\mathsf{rot}(\mat{B}^*)$, and so on. \todo{Maybe prove this better?}
There is also an isomorphism going on here that needs to be shown.
$\mathsf{rot}(\mat{Q}) = \mathsf{rot}(\mat{B}^* \mat{B}) = \mathsf{rot}(\mat{B}^*) \mathsf{rot}(\mat{B}) = \mathsf{rot}(\mat{B})^T \mathsf{rot}(\mat{B})$ due to $\mathsf{rot}()$ being an isomorphism.

The secret key $\mat{B}$ serves as a good basis for the private lattice in which a (hash digest of a) message will be signed in the signature generation step.
A simplified version of the key generation is described in the following algorithm:
\begin{algorithm}[H]\label{Simplified Hawk Key Generation}
\caption{Simplified Hawk Key Generation}
\begin{algorithmic}[1]
    \State Sample $f, g \in \bb{Z}[X] / (X^n + 1)$ from $\mathsf{bin}(\eta)$
    \State Compute $F, G$ s.t. $fG - Fg = 1 \mod \bb{Z}[X] / (X^n + 1)$ holds
    \State $\mat{B} \gets \begin{bmatrix} f & F \\ g & G \end{bmatrix}$
    \State $\mat{Q} \gets \mat{B}^* \mat{B}$
    \State \Return private key $\mat{B}$, public key $\mat{Q}$
\end{algorithmic}
\end{algorithm}
Polynomials $f$ and $g$ are sampled from a centered binomial distribution with $\eta$ depending on Hawk degree (see table \ref{tab:hawk-parameters})

\subsection{Solving the NTRU-equation}
An important point in the key generation process is finding proper polynomials $F$ and $G$ that satisfy the NTRU-equation $fG - Fg = 1 \mod \bb{Z}[X] / (X^n + 1)$.
Importantly, coefficients of $F$ and $G$ need to be rather small, and the procedure of solving the equation is a time-consuming part of the key generation process. 
\todo{Need some explanation of resultants here}
In the original NTRU system (both encryption and signature) resultants are used. Hawk uses the method proposed by Pornin and Prest in \cite{PP19} which improves upon the existing
resultant method in terms of both time- and memory complexity.

\subsection{Discrete Gaussian Distribution}
Before introducing the Hawk signature generation we describe the procedure to sample from the Discrete Gaussian Distribution, as described in \cite{HawkSpec24}. \\
Denote by $\dgd_{2\bb{Z} + c, \sigma }$ the Discrete Gaussian Distribution with parameter $c$ and $\sigma$.
Let $\rho_{\sigma}: \bb{Z} \rightarrow \bb{R}, \rho_{\sigma}(x) = \exp(\frac{-x^2}{2\sigma^2})$.
The Probability Density Function (PDF) of $\dgd_{2\bb{Z} + c, \sigma }$ is defined as:
\[
    \text{Pr[X = x] } = \frac{\rho_{\sigma}(x)}{\mathlarger{\sum_{y \in 2\bb{Z} + c} \rho_{\sigma}(y)}}
\]
For $z \geq 0$ we define the function $P_c(z) = \text{Pr}[ |X| \geq z ]$ when $X \sim \dgd_{2\bb{Z}+c, \sigma}$.
Using this function for $c \in \{0, 1\}$, one computes two Cumulative Distribution Tables (CDT) $T_0$ and $T_1$
such that $T_0[k] = P_0(2+2k)$ and $T_1[k] = P_1(3+2k)$ where $T_c[k]$ denotes the $k$-th index in the table.
In practice, to avoid using floating point numbers, the entries in the table are scaled by $2^{78}$ such that 
entries are integers with a high enough precision. For this overview and later theoretical analysis, however, we only consider the unscaled version.
Using the tables, we can define a procedure $\mathsf{sample}$ given by the following algorithm:

\begin{algorithm}[H]\label{sample}
% \caption{Simplified, unscaled sampling from $\dgd_{2\bb{Z}+c, \sigma}$ }
\caption{$\mathsf{sample}$}
\begin{algorithmic}[1]
    \Require{parameter $c \in \{0, 1\}$}
    \State $q \gets$ uniformly random from $[-1, 1] \in \bb{Q}$
    \State $z \gets 0$
    \State $v \gets 0$
    \While{$T_c[z] \neq 0$}
    \If{$|q| \leq T_c[z]$}
    \State $v \gets v + 1$
    \EndIf
    \State $z \gets z + 1$
    \State $v \gets 2v + c$
    \If{$q < 0$}
    \State $v \gets -v$
    \EndIf
    \EndWhile
    \State \Return $v$
\end{algorithmic}
\end{algorithm}

This algorithm only samples one point given parameter $c$.
We can extend this to generate vectors of length $n$ where each entry is distributed according to $\dgd_{2\bb{Z}+c, \sigma}$ by the following procedure:
\begin{algorithm}[H]
\caption{Sample vector of length n according to $\dgd_{2\bb{Z}+c, \sigma}$}
\begin{algorithmic}[1]
    \Require{Binary vector $\vec{t}$ with entries uniformly distributed from $\{0,1\}$}
    \State $\vec{x} \gets$ empty vector of length $2n$
    \For{$i = 0,1,..., 2n -1$}
    \State $\vec{x}[i] \gets \mathsf{sample}(\vec{t}[i])$
    \EndFor
    \Return $\vec{x}$
\end{algorithmic}
\end{algorithm}

In this manner, one can sample a lattice point $\vec{x}$ which is relatively close to a target lattice point $\vec{t}$, which will be used in the signature generation step.

\subsection{Hawk signature generation}
To generate a Hawk signature of a message $\vec{m}$, one computes a hash digest $\vec{h}$ of the message, and samples a point $\vec{x}$ from $\dgd_{2\bb{Z}^{2n} + \vec{t}, \sigma_{sign}}$ that is close to $\vec{h}$ in the 
private lattice generated by $\mathsf{rot}(\mat{B})$. By transforming the point $\vec{x}$ back to the lattice $\bb{Z}^{2n}$ via $\mat{B}^{-1}$ one has a signature that can be verified 
by anyone that has access to the public key $\mat{Q}$.
Below is the (simplified) procedure formulated in Algorithm \ref{simple_hawksign}:
\begin{algorithm}
    \caption{Simplified Hawk Signature Generation}\label{simple_hawksign}
\begin{algorithmic}[1]
    \Require{Message $\vec{m}$, private key $\mat{B}$}
    \State $M \gets H(\vec{m})$ \Comment{H is a hash function}
    \State $\vec{h} \gets H(M || \mathsf{salt})$ \Comment{$\mathsf{salt}$ is a randomly generated value and $||$ denotes concatenation}
    \State $\vec{t} \gets \mat{B} \vec{h} \mod 2$ \Comment{Reduction $\mod 2$ is done entrywise}
    \State $\vec{x} \gets \mathsf{sample}(\vec{t})$
    \If{$\lVert \vec{x} \rVert ^2 \geq 4 \cdot \sigma_{\text{sign}}^2 \cdot 2n$}
    \State restart
    \Else
    \State $\vec{w} \gets \mat{B}^{-1} \vec{x}$
    \State $\vec{s} \gets \frac{1}{2}(\vec{h} - \vec{w})$
    \State \Return signature $\vec{s}$, salt $\mathsf{salt}$
    \EndIf
    % \State restart if $\lVert \vec{x} \rVert ^2$ is too high
\end{algorithmic}
\end{algorithm}
\subsection{Hawk signature verification}
To verify a Hawk signature, one simply recomputes the vector $\vec{h}$ from $\vec{m}$ and $\mathsf{salt}$ and in turn the vector $\vec{w}$, and check if the $\mat{Q}$-norm of $\vec{w}$,
$\lVert \vec{w} \rVert_{\mat{Q}}^2$ is not too high. A signature with $\mat{Q}$-norm of appropriately short length will be considered valid.
A summary of the simplified procedure is given in Algorithm \ref{simple_hawkver}:

\begin{algorithm}[H]
    \caption{Simplified Hawk Signature Verification}\label{simple_hawkver}
\begin{algorithmic}[1]
    \Require{Signature $\vec{s}$, message $\vec{m}$, salt $\mathsf{salt}$, public key $\mat{Q}$ }
    \State $M \gets H(\vec{m})$ 
    \State $\vec{h} \gets H(M || \mathsf{salt})$
    \State $\vec{w} \gets \vec{h} - 2\vec{s}$
    \If{$\lVert \vec{w} \rVert_{\mat{Q}}^2 \geq 4 \cdot \sigma_{\text{verify}}^2 \cdot 2n$}
    \State \Return false
    \Else \ \Return true
    \EndIf
\end{algorithmic}
\end{algorithm}

We now briefly show why verification works: \\
As defined, 
\[
    \lVert \vec{w} \rVert_{\mat{Q}} = \sqrt{\vec{w}^* \mat{Q} \vec{w}} \implies \lVert \vec{w} \rVert_{\mat{Q}} ^2 = |\vec{w}^* \mat{Q} \vec{w}| \\
\]
\[
    \vec{w}^* \mat{Q} \vec{w} = \vec{w}^* \mat{B}^* \mat{B} \vec{w} = (\mat{B} \vec{w})^* (\mat{B} \vec{w}) = \lVert \mat{B} \vec{w} \rVert ^2
    = \lVert \vec{x} \rVert ^2
\]
\[
    \text{since } \vec{w} = \mat{B}^{-1} \vec{x} \implies \lVert \mat{B} \vec{w} \rVert ^2 = \lVert \mat{B} \mat{B}^{-1} \vec{x} \rVert ^2 = \lVert \vec{x} \rVert ^2
\]
This shows that by knowing $\mat{Q}$, one can measure the length of vectors in the secret lattice generated by $\mat{B}$. To find vectors close to a target vector $\vec{t}$ 
in the private lattice, however, requires knowledge about $\mat{B}$.
% In conclusion, a signature is only valid if the length of $\vec{x}$ is appropriately short in the lattice generated by $\mat{B}$, which requires knowledge of $\mat{B}$.
\subsection{Hawk security}
\section{Implementation of Hawk}
\subsection{Smaller degree Hawk}
For cryptanalysis and experiments, it can often be useful to have a "toy" version of the crypto-system at hand that has small parameters. Attacks against real life schemes may take days, even with good hardware and carefully implemented code. \todo{Find an anecdote here}
With smaller parameters, however, one can more easily check if a method for cryptanalysis is a viable candidate for further experimentation. Conversely, if a specific method appears useless facing a small-parametered scheme, it will most certainly
also be useless against the real life version of the scheme.

With this in mind, we need a small parameter version of Hawk.
Key generation is simple for any degree which is a power of 2, since the NTRU-solve method works on any power of 2.
Signature generation is also trivial to achieve. We don't consider signature verification for smaller parameters, since we are only interested in observng $\vec{w} = \mat{B}^{-1} \vec{x}$

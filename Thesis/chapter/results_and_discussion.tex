\chapter{Results and discussion}

\section{Results}

The method has unfortunately not proven to work, as no correct key has been found in any of the runs. It seems that regardless of number of signatures (above a certain point, e.g. one million), the method cannot give candidate solutions with
better comparison than random guessing. Random guessing in this case is assuming one knows what type of distribution the columns of a secret key is. One knows the distribution that $f, g$ follows, but $F$ and $G$ depends on $f$ and $g$.

For reference, I ran a test using the closeness measure $ \mathsf{diff_{min}} = \mathsf{min} \{\lVert \lvert \vec{b}' \rvert - \lvert \vec{b}_i \rvert \rVert \ \mathbf{:} \ \vec{b}_i \in \pm \mat{B}^{-1}\}$ by fixing one private key $(\mat{B}^{-1})$, 
and generating random keys $\mat{B}^{'-1}$ (which will serve as random guesses), to check if the attack on average gives better results than random
guessing. Table 1 shows the result of comparing a key with 100 random keys, and the result of 100 random starting points for the gradient search (both ascent and descent).

One thing to note is that Hawk does not specify parameters (such as width $\sigma$ of $\dgdi$) for lower values of $n$ than 256. Therefore, when sampling signatures for $n=32, 64$ and $128$, I use the same tables, i.e. same $\sigma$ for $\dgdi$, as in Hawk 256.

\begin{table}[H]
    \centering
    \caption{Closeness measure for Hawk attack}
    % \label{tab:hawk-parameters}
    \begin{tabular}{lcccc}
        \toprule
        \textbf{Type} & $\mathsf{diff_{min}}$ & $\mathsf{diff_{max}}$ & \textbf{Avg $\mathsf{diff_{min}}$} & \textbf{Avg $\mathsf{diff_{max}}$} \\
        \midrule
        Key comparison degree 32 & 6.25 & 15.81 & 7.74 & 11.22 \\
        Attack on Hawk 32 (1m samples) & 7.14 & 16.24 & 8.50 & 12.87 \\
        \midrule
        Key comparison degree 64 & 10.77 & 26.51 & 13.96 & 22.18 \\
        Attack on Hawk 64 (1m samples) & 13.49 & 26.00 & 16.25 & 22.59 \\
        \midrule
        Key comparison degree 128 & 25.33 & 47.26 & 30.00 & 41.60 \\
        Attack on Hawk 128(1m samples) & 24.27 & 46.91 & 29.61 & 46.91 \\
        \midrule

        Key comparison degree 256 & 56.33 & 82.34 & 61.02 & 75.02 \\
        Attack on Hawk 256 (1m samples) & 54.50 & 84.07 & 59.20 & 71.09 \\ 
        Attack on Hawk 256 (10m samples) & 57.72 & 77.54 & 62.37 & 77.54 \\
        \bottomrule
    \end{tabular}
\end{table}
Lastly, I include values from sampling points from $\dgdi$ for Hawk 256. 100 million vectors $\vec{x}$ were sampled, resulting in $2 \cdot 256 \cdot 100$ million independently sampled points from $\dgdi$.
First, I compute $\sigma^2$ and $\sigma$, then compute $\mu_4$ of normalized samples on the form $z = \frac{x}{\sigma}$.
We then get that 
\begin{itemize}
    \item $\sigma^2 = 4.080429335$
    \item $\sigma = 2.020007261$
    \item Normalized $\mu_4 = \bb{E} (z^4) = 3.000007624$
    \item $(\mu_4 - 3) = 0.000007624$
\end{itemize}

\section{Conclusion}
We have computed $\mu_4$, i.e. the 4th moment of Z to be 2.999..., and that the difference ($\mu_4 - 3$) is about $2.1 \cdot 10^{-14}$ for parameter 256.
Let us now compute how many samples one would need to successfully distinguish $mom_{4, \mat{C}}(\vec{u})$ when $\vec{u} \in \pm \mat{C}$ and when $\vec{u} \not\in\pm\mat{C}$.
$mom_{4, \mat{C}}(\vec{u}) = \bb{E}[(\langle \vec{u}, \mat{C}\vec{x} \rangle ^4] = 3 + (\mu_4 - 3) \mathlarger{\sum_{i=1}^{n} \langle \vec{u}, \vec{c}_i \rangle ^4}$.
If $\vec{u} = \vec{c}_i$, the entire expression is $3 + (\mu_4 - 3)\cdot 1 = \mu_4$ because $\langle \vec{c}_i, \vec{c}_i \rangle = 1$ and $\langle \vec{c}_i, \vec{c}_j \rangle = 0$ for $i \neq j$. 
Otherwise, the sum $(\mathlarger{\sum_{i=1}^{n} \langle \vec{u}, \vec{c}_i \rangle ^4}) < 1$.


Say one wants to distinguish $mom_{4, \mat{C}}(\vec{c}_i)$ and $mom_{4, \mat{C}}(\vec{u})$ for $\vec{u} \not\in\pm\mat{C}$.
Let $y = \langle \vec{u}, \mat{C} \vec{x} \rangle$ and assume $y$ follows a standard normal distribution with $\mu = 0$ and $\sigma^2 = 1$.
Then we approximate the variance $\sigma^2$ and standard deviation $\sigma$ for $\langle \vec{u}, \mat{C}\vec{x} \rangle ^4$ as $\bb{E}[y^8] - \bb{E}[y^4]^2$.
Then $\sigma^2 \approx 105 - 9 = 96$ and $\sigma = \sqrt{96} \approx 9.79796$.

Now, to estimate number of samples we use formula for confidence intervals and Central Limit Theorem. Let $s$ denote required number of samples. Then 
\[
    s \geq \mathlarger{(\frac{Z_{\alpha/2} \cdot \sigma}{\mathsf{err}})^2}
\]
Now, we want $\mathsf{err}$ to be smaller than $2.1 \cdot 10^{-14}$ for parameter 256.
If we set $\mathsf{err}$ to be $10^{-14}$, and we want to determine a difference with, say, 75 \% confidence, we get:

\[
    s \geq (\frac{0.68 \cdot 9.79796}{10^{-14}})^2 \approx 4.439 \cdot 10^{29}
\]
where $0.68$ is from standard normal tables for 75 \%.
This result is way beyond the limit of acceptable transcript size, which for parameter 256 is $2^{32} \approx 4.3 \cdot 10^{9}$ and for non-challenge parameters 512 and 1024 is $2^{64} \approx 2 \cdot 10^{19}$.
One can therefore conclude that the attack will not be successful.

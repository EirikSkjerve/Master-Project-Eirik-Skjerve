\newcommand{\PP}[2][]{\mathcal{P}_{#1}(\mat{#2})}
\newcommand{\mat}[1]{\mathit{#1}}
\renewcommand{\vec}[1]{\mathbf{#1}}
\newcommand{\GLnR}{\mathcal{GL}_{n}(\mathbb{R})}
\newcommand{\normdist}[2]{\mathcal{N}(#1, #2)}
\newcommand{\bb}[1]{\mathbb{#1}}

\chapter{Hawk and \textit{Learning a parallelepiped}}

\section{Hawk}
In the following Hawk, the digital signature scheme, will be presented, as in \cite{HawkSpec24}
\subsection{Simple Hawk}
Present simple sketch of keygen, sign and ver, as well as an example in dimension 2
\subsection{Key generation}
\subsection{Signature generation}
\subsection{Signature verification}

\section{Learning a parallelepiped}
The paper \textit{Learning a Parallelepiped: Cryptanalysis of GGH and NTRU Signatures} by Phong Q. Nguyen and Oded Regev from 2006 
introduced a method for breaking digital signature schemes based on the GHH scheme \cite{NR09}. 
Essentially, by observing enough \textit{message, signature} pairs generated by a secret key one can deduce this secret key. 
The attack broke the NTRU-Sign scheme by observing as little as 400 signatures.

\subsection{Solving the Hidden Parallelepiped Problem}
First we define an idealized version of both the problem to solve and the solution as proposed in \cite{NR09}. Then we discuss the problem and solution in a practical context.
\paragraph{Hidden Parallelepiped Problem (HPP):} Let $\mat{V} = [\vec{v}_1, ... \vec{v}_n ]$ be a secret $n \times n$ matrix and $\mat{V} \in \GLnR$.
Define by $\PP{V} = \{\sum_{i=1}^{n} x_i \vec{v}_i : x_i \in [-1, 1]\}$ a secret $n$-dimensional parallelepiped defined by V.
Given a polynomial (in $n$) number of vector samples uniformly distributed over $\PP{V}$, recover the rows $\vec{v}_i$ of $\mat{V}$. \\
We solve this problem in two main steps: 
\begin{enumerate}
    \item Transforming the hidden parallelepiped into a hidden hypercube:
    \item Learning the hypercube:
\end{enumerate}
\paragraph{Parallelepiped $\rightarrow$ Hypercube}
The first step of the attack is to convert our hidden parallelepiped $\PP{V}$ into a hidden hypercube 
$\PP{C}$ with orthogonal basis vectors.
Essentially, one moves each sampled point in accordance to a transformation matrix $\mat{L}$ computed the following way: \\
\begin{itemize}
    \item Approximate $\mat{G} \approx \mat{V}^T \mat{V}$ using our samples $\mathcal{X} = \{\vec{x}_1, ..., \vec{x}_u\}$
    \item Compute $\mat{L}$ such that $\mat{L} \mat{L}^{T} = \mat{G}^{-1}$
    \item Then $\mat{C} = \mat{V} \mat{L}$
    \item By multiplying our samples $\mathcal{X}$ to the right by $\mat{L}$, they are now uniformly distributed over the hidden hypercube $\PP{C}$
    \item \textbf{(---SHOW CALCULATIONS---)}
\end{itemize}

\paragraph{Learning a Hypercube:}
The second step is to learn the hypercube. Given samples over $\PP{C}$, we deduce the rows of the secret matrix $\mat{C}$ with the method described in Algorithm 1. After the rows are approximated,
one can multiply the rows $\{\vec{c}_1, ..., \vec{c}_2\}$ by $\mat{L}^{-1}$ such that we have $\{\vec{v}_1, ..., \vec{v}_2\}$, and we are done.
\begin{algorithm}
    \caption{Learning a Hypercube}
    \begin{algorithmic}
        \Require Descent parameter $\delta$, samples $\mathcal{X}$ uniformly distributed over $\PP{C}$
        \Ensure A row vector $\pm \vec{v}_i$ of $\mat{C}$
        \State Choose uniformly at random $\vec{w}$ on the unit sphere of $\bb{R}^n$
        \Loop
        \State Compute $\vec{g}$, an approximation of $\nabla mom_{4}(\vec{w})$
        \State Let $\vec{w}_{new} = \vec{w} - \delta \vec{g}$
        \State Place $\vec{w}_{new}$ back on the unit sphere by dividing it by $\left \Vert \vec{w}_{new} \right \Vert$
        \If{$mom_4(\vec{w}_{new}) \ge mom_4(\vec{w})$} \Comment $mom_4$ are approximated by samples 
            \State \Return $\vec{w}$
        \Else
            \State Replace $\vec{w}$ with $\vec{w_{new}}$ and continue loop
        \EndIf
        \EndLoop
    \end{algorithmic}
\end{algorithm}
\hfill \break \\
\subsection{HPP against NTRU}
\subsection{HPP against normally distributed samples}
In the following, we see what happens to the computations that the attack is based on if we replace the uniform distribution by a normal distribution.
The key component and assumption of the \textit{Learning a parallelepiped} attack is that the provided samples are distributed \textit{uniformly} over $\PP{V}$.
Recall that $\PP{V}$ is defined as $\{\sum_{i=1}^n x_i \vec{v}_i : x_i \in [-1, 1]^n\}$ where $\vec{v}_i$ are rows of $\mat{V}$. (Generally one could take another interval than $[-1, 1]$ and do appropriate scaling.)
It is clear that one runs into trouble if the sampled vectors are on the form $\vec{v} = \vec{x} \mat{V}$ where $\vec{x}$ follows a normal distribution, i.e. $x_i \sim \normdist{\mu}{\sigma}$.

\paragraph{Part 1}
First of all, the distribution $\normdist{\mu},{\sigma}$ is defined over the interval $[- \infty, \infty]$, so it does not make sense to talk about samples "normally distributed over $\PP{V}$" without tweaking any definitions.
Therefore, let $[- \eta, \eta]$ be a finite interval on which to consider a truncated normal distribution $\normdist{\mu}{\sigma})$ such that $\int_{-\eta}^{\eta} f_X(x) dx = 1 - \delta$ for some small $\delta$
where $f_X(x)$ is the probability density function of $\normdist{\mu}{\sigma}$ (see definition in section 2).
Now we consider $\PP[\eta]{V} = \{\sum_{i=1}^n x_i \vec{v}_i : x_i \in [-\eta, \eta]^n\}$ and proceed as in the original HPP with $\PP[\eta]{V}$ instead of $\PP{V}$.

\paragraph{Approximating $\mat{V}^t \mat{V}$}
Let $\mat{V} \in \GLnR$. Let $\vec{v}$ be chosen from a truncated normal distribution $\normdist{0}{\sigma}$ over $\PP[\eta]{V}$.
Then $\lim_{\eta\to\infty}$ $\bb{E}[\vec{v}^t\vec{v}] = \mat{V}^t \mat{V} \cdot \sigma^2$, where $\sigma^2$ is the variance of the current distribution.

\begin{proof}
    Let $\vec{v} = \vec{x}\mat{V}$ be all our samples, where $\vec{x}$ is such that $x_i \sim \normdist{0}{\sigma}$ over the interval $[-\eta, \eta]$.
Then $\vec{v}^t\vec{v} = \mat{V}^t \vec{x}^t \vec{x} \mat{V}$. Considering $\bb{E}[\vec{x}^t \vec{x}]$, we see that for $i \neq j$, 
$\bb{E}[x_i x_j] = \bb{E}[x_i] \bb{E}[x_j] = 0 \cdot 0 = 0$ due to independent random variables.
For $i=j$, $\lim_{\eta\to\infty} \bb{E}[x_i^2] = \bb{V}[x_i]$ since $\bb{V}[x_i] = \bb{E}[x_i^2] - \bb{E}[x_i]^2 = \bb{E}[x_i^2] - 0 = \sigma ^2$.
Therefore, $\lim_{\eta\to\infty} \bb{E}[\vec{x}^t \vec{x}] = \mat{I}_n \cdot \sigma^2$, i.e. the identity matrix with diagonal entries multiplied by $\sigma ^2$.
Consequently, $\lim_{\eta\to\infty} \vec{v}^t \vec{v} = \mat{V}^t \bb{E}[\vec{x}^t\vec{x}] \mat{V} = \mat{V}^t \mat{I}_n \cdot \sigma^2 \mat{V} = (\mat{V}^t \mat{V}) \cdot \sigma ^2$ 
and conversely $\lim_{\eta\to\infty} \mat{V}^t \mat{V} = (\vec{v}^t \vec{v})/ \sigma^2$.
\end{proof}

This means that we can in theory approximate the covariance matrix $\mat{V}^t \mat{V}$ by averaging over $\vec{v}^t \vec{v}$ and dividing by $\sigma ^2$. 
However, experiments show that a normal distribution requires many more samples, perhaps exponentially many(?). Why is this??
The question is also the following: how accurate does our approximation actually need to be?
\paragraph{Hypercube transformation}
Assume now that we know $\mat{V}^t \mat{V}$. Consider instead of $\PP{V}$, $\PP[\eta]{V}$.
Then by following part 1 of \textbf{Lemma 2} and its proof from \cite{NR09} we can transform our hidden parallelepiped $\PP[\eta]{V}$ into $\PP[\eta]{C}$, a hidden hypercube,
since this does not depend on what distribution the samples follow - it only assumes one knows $\mat{V}^t\mat{V}$.
For completeness, by adapting the second part of the proof of \textbf{Lemma 2} to our case: 
\begin{proof}
    Let $\vec{v} = \vec{x}\mat{V}$ where $\vec{x}$ is normally distributed according to $\normdist{0}{\sigma}$, however we only consider the finite interval $[-\eta,\eta]$.
    It follows then that $\vec{v}\mat{L} = \vec{x}\mat{V}\mat{L} = \vec{x}\mat{C}$ has a (truncated) uniform distribution over $\PP[\eta]{C}$.
\end{proof}
Thus, we should also, in theory, be able to map our normally distributed samples onto the hidden hypercube.

\paragraph{Learning a hypercube}
It is clear that samples uniformly over $\PP[\eta]{V}$ centered at the origin will form a hypersphere of which every rotation leaves the sphere practically equal. As a consequence, any measure of the fourth moment of one-dimensional projections are 
useless because every projection will yield the same result.
For completeness, we show the results.


% As a countermeasure against these types of attacks, it was proposed in \cite{GPV07} to not use the uniform distribution when creating signatures. Rather, by sampling from a Discrete Gaussian distribution
% (a discrete analogue to the Normal distribution), signatures do not "leak" any information about the secret parallelepiped, and the geometry of the key remains concealed.

\chapter{Hawk and \textit{Learning a parallelepiped}}

\section{Hawk}
In the following Hawk, the digital signature scheme, will be presented, as in \cite{hawkspec}
\subsection{Simple Hawk}
Present simple sketch of keygen, sign and ver, as well as an example in dimension 2
\subsection{Key generation}
\subsection{Signature generation}
\subsection{Signature verification}

\section{Learning a parallelepiped}
The paper \textit{Learning a Parallelepiped: Cryptanalysis of GGH and NTRU Signatures} by Phong Q. Nguyen and Oded Regev from 2006 
introduced a method for breaking digital signature schemes based on the GHH scheme \cite{hpp}. 
Essentially, by observing enough \textit{message, signature} pairs, one can deduce the secret key.

By collecting enough signatures on the form $\mathbf{s} = \nint{\textbf{m} \mathbf{B}^{-1}}\mathbf{B}$
where $\mathbf{m}$ is a hash of some message and $\mathbf{B}$ is the secret basis, one can recover $\mathbf{B}$. 
\subsection{Assumptions}
First we look at an idealized case to get some understanding of the Hidden Parallelepiped Problem:
Let $\mathbf{B}$ be a secret $n \times n$ matrix. Let $\{\mathbf{s}_1, \mathbf{s}_2, ... \mathbf{s}_p\}$ where $\mathbf{s}_i = \nint{\textbf{m}_i \mathbf{B}^{-1}}\mathbf{B}$
and $\mathbf{m}_i$ is uniformly distributed over some interval $[0, q]$ be $p$ signatures on "random" messages.
\subsection{Covariance matrix}
\subsection{Hidden parallelepiped to hidden hypercube}
\subsection{Gradient descent}
\subsection{Example in dimension 2}
\subsection{Hawk resistance against HPP}

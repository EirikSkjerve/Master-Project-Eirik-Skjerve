\chapter{Introduction}
\section{Context and motivation}
Digital signatures are an integral part of secure communication today. The widely used DSA (Digital Signature Algorithm)
and RSA (Rivest, Shamir, and Adleman) signature schemes are in peril due to the potential emergence of quantum computers which, 
theoretically, are able to break the hard problems DSA and RSA-sign are based upon.
Whether practical quantum computers with these powers will emerge any time soon is debatable. However, measures against this potential looming threat has already begun. 
In 2016, NIST (National Institute of Standards and Technology) announced a process for selecting new standard schemes for Key Encapsulation Methods (abbr. KEMs) and 
digital signatures that are resilient against quantum computers (source to NIST pages here). Many of the submissions to this process (including KRYSTALS-Dilithium which is to be standardized) 
are based on lattice problems that are believed to be hard to solve for both classical and quantum computers. \hfill \break \\

Cryptographic schemes based on lattice problems are not an enirely new phenomenon, however. The NTRU scheme and its signature counterpart NTRU-Sign, published in 2003 (source for this),
is a digital signature scheme based on the hardness of the Closest Vector Problem.
The original scheme was broken by Phong. Q. Nguyen \& Oded Regev in 2006 \cite{NR09}, who showed that one can retrieve a secret key by observing enough signatures generated with the key.
A newer digital signature scheme submitted to NIST's standardization process, Hawk \cite{HawkSpec24}, is somewhat similar to that of NTRU. 

\section{Objectives}
The objective for this thesis consists of two main parts:
\begin{itemize}
\item \textbf{Implementation of Hawk in Rust}. As the first part of the thesis I implement the Hawk digital signature scheme in the Rust programming language. 
    Implementing a scheme on ones own is a good way to actually learn how it works. I chose to implement it in Rust for the sake of learning the programming language.
    Moreover, having ones own version makes it easier to experiment, adjust and modify to ones need. It would also be a challenge to understand and work with complicated source code someone else has written.

\item \textbf{Cryptanalysis and experimentation}. As part two of the thesis I want to do cryptanalysis of Hawk. The goal is to use the "Learning a parallelepiped" attack \cite{NR09} and adjusting it to try and break Hawk. 
    This requires both theoretical and practical work, and experiments will be implemented in Rust.
\end{itemize}
\section{Thesis outline}
Chapter 2 will introduce important notions and mathematical background used in this thesis. Chapter 3 will introduce Hawk and the \textit{Learning a Parallelepiped} attack and implementations of these. 
In Chapter 4, a version HPP attack aimed at Hawk will be presented. Chapter 5 will show results, and Chapter 6 will discuss future work.

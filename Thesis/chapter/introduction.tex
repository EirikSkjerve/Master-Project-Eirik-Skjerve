\chapter{Introduction}
\section{Context and motivation}
Digital signatures are an integral part of secure communication today. The widely used DSA (Digital Signature Algorithm)
and RSA (Rivest, Shamir, and Adleman) signature schemes are in peril due to the potential emergence of quantum computers which, 
theoretically, are able to break the hard problems DSA and RSA-sign are based upon.
Whether practical quantum computers with these powers will emerge any time soon is debatable. However, measures against this potential looming threat has already begun. 
In 2016, NIST (National Institute of Standards and Technology) announced a process for selecting new standard schemes for Key Encapsulation Methods (abbr. KEMs) and 
digital signatures that are resilient against attacks based on quantum computers (source to NIST pages here). Many of the submissions to this process (including KRYSTALS-Dilithium which is to be standardized) 
are based on lattice problems that are believed to be hard to solve for both classical and quantum computers. \hfill \break \\

Cryptographic schemes based on lattice problems are not an enirely new phenomenon, however. The NTRU scheme and its signature counterpart NTRU-Sign, published in 2003 (source for this),
is a digital signature scheme based on the hardness of the Closest Vector Problem.
The original scheme was broken by Phong. Q. Nguyen \& Oded Regev in 2006 \cite{NR09}, who showed that one can retrieve a secret key by observing enough signatures generated with one key.
In other words, each signature leaks some information about the secret key.
A newer digital signature scheme submitted to NIST's standardization process, Hawk \cite{HawkSpec24}, is somewhat similar to that of NTRU.
One key difference is that Hawk signatures allegedly do not leak any information about the secret key in this manner due to the signature generation process.

\section{Objectives}
The objective for this thesis consists of two main parts:
\begin{itemize}
    \item \textbf{Implementation of Hawk in Rust}. As the first part of this thesis I implement the Hawk digital signature scheme according to \cite{HawkSpec24} in the Rust programming language. 
    Implementing a scheme and its algorithms on ones own is a good way to learn how it works. I chose to implement it in Rust for the sake of learning this programming language as a bonus objective of the thesis.
    Moreover, having ones own version of an algorithm makes it easier to experiment, adjust and modify it to ones need. It would in any case be challenging to understand and work with dense, long, 
    and complicated source code someone else has written. (For the Hawk teams source code and reference implementation see https://github.com/hawk-sign )

\item \textbf{Cryptanalysis and experimentation}. The second part of this thesis is cryptanalysis of Hawk. The goal is to use the \textit{Learning a parallelepiped} attack \cite{NR09} and adjusting it to try and break Hawk. 
    This requires both theoretical and practical work, and experiments will, like the Hawk implementation itself, be implemented in Rust.
\end{itemize}
\section{Thesis outline}
Chapter 2 will introduce important notions and mathematical background used in this thesis. Chapter 3 will introduce Hawk and the \textit{Learning a Parallelepiped} attack, as well as discussing the implementation of these.
In Chapter 4 our cryptanalysis of Hawk is presented. Chapter 5 will show results, and Chapter 6 will discuss future work.

\chapter{Introduction}
\section{Context and motivation}
Digital signatures are an important part of secure communication today. The most used cryptographic scheme used for digital signatures today is DSA (Digital Signature Algorithm) or RSA (Rivest, Shamir, and Adleman) signatures (source).
However, in 1994, Peter Shor developed Shor's algorithm, which, given a large enough quantum computer, is able to solve the hard problems DSA and RSA is based upon, namely the Discrete Logarithm Problem and Prime Factorization(source). 
Whether big enough quantum computers will emerge any time soon is debatable. However, measures against this potential looming threat has already begun. In 2016, NIST (National Institute of Standards and Technology)
announced a standardization process for new standard schemes for KEMs (Key Encapsulation Methods) and digital signatures that have strong security against quantum computers (source). Many of the submissions to this process,
including KRYSTALS-Dilithium which is to be standardized, are based on lattice problems that are believed to be hard to solve for both classical and quantum computers (source). \hfill \break \\

Cryptographic schemes based on lattice problems are not an enirely new phenomenon, however. NTRU-Sign, published in 2003(source), is a digital signature scheme based on the hardness of the Closest Vector Problem (source).
The original scheme was broken due to Phong. Q. Nguyen \& Oded Regev in 2006 \cite{hpp}, who showed that by observing enough signatures generated with one secret key, one can retrieve the secret key.
A newer digital signature scheme, Hawk (source), submitted to NIST's standardization process, is a scheme similar that of NTRU and GGH. 
The goal of this thesis is to try and adapt the Hidden Parallelepiped Problem attack to Hawk \cite{hawk}. Hawk spec here \cite{hawkspec}

\section{Objectives}
The objective for this thesis consists of two main parts:
\begin{itemize}
\item \textbf{Implementation of Hawk in Rust}. As the first part of the thesis I implement the Hawk digital signature scheme in the Rust programming language. 
    Implementing a scheme on ones own is a good way to actually learn how it works. I chose to implement it in Rust for the sake of learning the programming language.
    Moreover, having ones own version makes it easier to experiment, adjust and modify to ones need. It would also be a challenge to understand and work with complicated source code someone else has written.

\item \textbf{Cryptanalysis and experimentation}. As part two of the thesis I want to do cryptanalysis of Hawk. The goal is to use the "Learning a parallelepiped" attack \cite{hpp} and adjusting it to try and break Hawk. 
    This requires both theoretical and practical work, and experiments will be implemented in Rust.
\end{itemize}
\section{Thesis outline}
Chapter 2 will introduce important notions and mathematical background used in this thesis. Chapter 3 will introduce Hawk and the \textit{Learning a Parallelepiped} attack and implementations of these. 
In Chapter 4, a version HPP attack aimed at Hawk will be presented. Chapter 5 will show results, and Chapter 6 will discuss future work.

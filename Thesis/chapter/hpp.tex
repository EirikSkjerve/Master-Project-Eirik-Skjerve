\newcommand{\PP}[2][]{\mathcal{P}_{#1}(\mat{#2})}
\newcommand{\mat}[1]{\mathit{#1}}
\renewcommand{\vec}[1]{\mathbf{#1}}
\newcommand{\GLnR}{\mathcal{GL}_{n}(\mathbb{R})}
\newcommand{\normdist}[2]{\mathcal{N}(#1, #2^2)}
\newcommand{\dgdist}{\mathcal{D}_{2\bb{Z}+c, \sigma}}
\newcommand{\bb}[1]{\mathbb{#1}}

\newtheorem{lemma}{Lemma}
\newtheorem{proof}{Proof}

\section{HPP}
In this section we present the \textit{Learning a Parallelepiped} attack as described in \cite{NR09}, with some different notation to not confuse with Hawk notation.
\subsection{Setup and idealized version}
Let $\mat{V} \in \mathcal{GL}(\bb{R})$ be a secret $n \times n$ unimodular matrix and $\PP{V}$ be a fundamental parallelepiped, defined as $\{V \vec{x}: \vec{x} \in [-1, 1]^n\}$. Let $\vec{x} = \{\vec{x}_1, \vec{x}_2, ...,\vec{x}_t\}$
be $t$ row vectors of length $n$ with entries uniformly distributed over $[-1, 1] \in \bb{Q}$ and $\vec{v} = \{\vec{v}_1, \vec{v}_2, ..., \vec{v}_t\} = \{\mat{V} \vec{x}_1, \mat{V} \vec{x}_2, ..., \mat{V} \vec{x}_t\}$ 
such that $\vec{v} \subset \PP{V}$. By observing $\vec{v}$ for large enough $t$, one is able to retrieve the rows of $\pm \mat{V}$ by the following steps:
\begin{enumerate}
    \item Estimate covariance matrix $\mat{V}^t \mat{V}$
    \item Transform samples $\vec{v} \in \PP{V}$ to $\vec{c} \in \PP{C}$ where $\PP{C}$ is a hypercube, i.e. $\mat{C} \mat{C}^t = \mat{I}$
    \item Do gradient descent to minimize the fourth moment of one-dimensional projections and reveal a row of $\pm \mat{C}$ which finally can be transformed into a row of $\pm \mat{V}$
\end{enumerate}
In the following, each of these steps will be covered in detail.
\subsection{Covariance matrix estimation}
Given enough samples $\vec{v} = \mat{V} \vec{x}$, we want to estimate the covariance matrix $\mat{G} \approx \mat{V}^t \mat{V}$.
This is achieved as $\vec{v}^t \vec{v} = (\mat{V} \vec{x})^t(\mat{V} \vec{x}) = \mat{V}^t \vec{x}^t \vec{x} \mat{V}$.
% Remark that since $\vec{x}$ is a row vector, $\bb{E}[\vec{x}^t \vec{x}]$ will be an $n \times n$ matrix.
Now, $\bb{E}[\vec{x}^t \vec{x}] = \mat{I} / 3$ where $\mat{I}$ is the identity matrix because of the following:
Since all $x_i \in \vec{x}$ are independent, and $\bb{E}[x_i] = 0$, we have that $\bb{E}[x_i x_j] = \bb{E}[x_i] \bb{E}[x_j] = 0$ for $i \neq j$.
For the case when $i = j$, we have \[\bb{E}[x_i x_j] = \bb{E}[x ^2] = \int_{a}^{b} x^2 \frac{1}{b-a} dx = \int_{-1}^{1} x^2 \frac{1}{2} dx = \frac{1}{3}\]
Therefore, $\bb{E}[\vec{x}^t \vec{x}]$ is $\frac{1}{3}$ down the diagonal and is 0 otherwise, i.e. $\mat{I} / 3$.
Consequently, $\vec{v} ^t \vec{v} = \mat{V}^t (\mat{I} / 3) \mat{V}$, and therefore $\vec{v}^t \vec{v} \cdot 3 = \mat{V}^t \mat{V}$.
Clearly, the more samples $\vec{v}$ one has, the more accurate the approximation $\mat{G} \approx \mat{V} ^t \mat{V}$ is.
\subsection{Hidden parallelepiped to hidden hypercube transformation}
Given a good approximation $\mat{G}$ of $\mat{V}^t \mat{V}$, the next step is to calculate a linear transformation $\mat{L}$ such that the following is true:
\begin{enumerate}
    \item $\mat{C} = \mat{V}\mat{L}$ is orthonormal, i.e. the rows are pairwise orthogonal and the norm of each row is 1. In other words, $\mat{C} \mat{C}^t = \mat{I}$. 
        Consequently, $\PP{C}$ becomes a hypercube.
    \item If $\vec{v}$ is uniformly distributed over $\PP{V}$ then $\vec{c} = \vec{v} \mat{L}$ is uniformly distributed over $\PP{C}$.
\end{enumerate}
This is achieved by taking the Cholesky decomposition of $\mat{G}^{-1} = \mat{L} \mat{L}^t$. 
To compute Cholesky decomposition of $\mat{G}^{-1}$, we must first show that $\mat{G}$ is symmetric positive definite.
$\mat{G}$ is symmetric $\iff \mat{G}^{t}= \mat{G}$ which is clear as $\mat{G}^t = (\mat{V}^t\mat{V})^t = \mat{V}^t (\mat{V}^t)^t = \mat{V}^t \mat{V} = \mat{G}$.
$\mat{G}$ is positive definite if for any non-zero column vector $\vec{x} \in \bb{R}^n$, $\vec{x}^t \mat{G} \vec{x} > 0.$
We have that $\vec{x}^t \mat{G} \vec{x} = \vec{x}^t \mat{V}^t \mat{V} \vec{x} = (\mat{V} \vec{x})^t (\mat{V} \vec{x})$. Denote by $\vec{y} = \mat{V} \vec{x}$.
Since $\vec{x} \neq \mathbf{0}$ and $\mat{V}$ needs is invertible and therefore non-zero, it is clear that 
$\vec{y} \neq \mathbf{0}$ and $\vec{y}^t \vec{y} = \lvert \vert{\vec{y}} \rvert \vert ^2 > 0$.

From this, we easily show the following:
\begin{enumerate}
    \item If $\mat{C} = \mat{V} \mat{L}$, then $\mat{C} \mat{C}^t = \mat{V} \mat{L} \mat{L}^t \mat{V}^t = \mat{V} \mat{G}^{-1} \mat{V}^t = \mat{V} (\mat{V^t}\mat{V})^{-1} \mat{V}^t
        = \mat{V} \mat{V}^{-1} \mat{V}^{-t} \mat{V}^t = \mat{I}$.
    \item Since entries in $\vec{x}$ is uniformly distributed, then $\vec{c} = \mat{C} \vec{x}$ is clearly uniformly distributed over $\PP{C}$.
\end{enumerate}

By multiplying our samples $\vec{v}$ by $\mat{L}$ on the right, we transform them from the hidden parallelepiped to the hidden hypercube. If one finds
the rows of $\pm \mat{C}$, one can simply multiply the result on the right by $\mat{L}^{-1}$ to obtain the solution for $\mat{V}$.

\subsection{Moments and Gradient Descent}
We now measure and minimize the fourth moment of one-dimensional projections to disclose rows of $\pm \mat{C}$.
Let $mom_{k, \mat{C}}(\vec{w})$ be defined as the $k$-th moment of $\PP{C}$ projected onto $\vec{w}$, i.e. 
$\bb{E}[\langle \vec{c}, \vec{w} \rangle ^k]$ where $\vec{c} = \vec{x} \mat{C}$ for uniformly distributed $\vec{x}$ and $\vec{w} \in \bb{R}^n$.
Looking at the term $\langle \vec{c}, \vec{w} \rangle$, we have $\langle \vec{x} \mat{C}, \vec{w} \rangle = \langle \sum_{i=1}^{n} x_i c_i, \vec{w} \rangle$
$ = \sum_{i=1}^{n} x_i \langle c_i, \vec{w} \rangle $.
Moving this inside the $\bb{E}[ \ ]$ we have $\bb{E}[\sum_{i=1}^{n} x_i \langle c_i, \vec{w} \rangle ^k ]$.
We look at the two cases when $k = 2$ and $k=4$.
\begin{itemize}
    \item $\mathbf{k = 2: }$
        $\bb{E}[(\sum_{i=1}^{n} x_i \langle c_i, \vec{w} \rangle ) ^2] = \bb{E}[\sum_{i}^{n} \sum_{j}^{n} x_i x_j \langle c_i, \vec{w} \rangle \langle c_j, \vec{w} \rangle]$.
        As seen before in section 3.3.2, $\bb{E}[x_i x_j] = \frac{1}{3}$ when $i = j$ and $0$ otherwise. Thus, we have the final expression 
        $mom_{2, C}(\vec{w}) = \frac{1}{3} \sum_{i}^{n} \langle c_i, \vec{w} \rangle ^2$ which can also be written as
        $ = \frac{1}{3} \vec{w} \mat{C}^t \mat{C} \vec{w}^t$
    \item $\mathbf{k = 4: }$
        \[ \bb{E}[(\sum_{i=1}^{n} x_i \langle c_i, \vec{w} \rangle ) ^4] = 
        \bb{E}[\sum_{i}^{n} \sum_{j}^{n} \sum_{k}^{n} \sum_{l}^{n} x_i x_j x_k x_l 
        \langle c_i, \vec{w} \rangle \langle c_j, \vec{w} \rangle
    \langle c_k, \vec{w} \rangle \langle c_l, \vec{w} \rangle] \]
    There are three cases for the indices $i, j, k$ and $l$:
    \begin{enumerate}
        \item \textbf{All equal: } If $i = j = k = l$, we simply have $\sum_i^n \bb{E}[x^4] \langle c_i, \vec{w} \rangle ^4 = \frac{1}{5} \sum_i \langle c_i, \vec{w} \rangle ^4$
            due to the fact that $\bb{E}[x ^4] = \int_{-1}^{1} x^4 \frac{1}{2} dx = \frac{1}{5}$
        \item \textbf{None equal:} If $i \neq j \neq k \neq l$ the expression is zero due to $\bb{E}[x_i] = 0$ and all $x_i, x_j, x_k$ and $x_l$ are independent.
        \item \textbf{Pairwise equal:} If either 
            \begin{itemize}
                \item $i = j \neq k = l$
                \item $i = k \neq j = l$ 
                \item $i = l \neq j = k$
            \end{itemize}
            then we have 
            $\sum_{i \neq j} \bb{E}[x_i ^2 x_j^2] \langle c_i, \vec{w} \rangle ^2 \langle c_j, \vec{w} \rangle^2$ 
            $ = \frac{1}{9} \sum_{i \neq j} \langle c_i, \vec{w} \rangle^2 \langle c_j, \vec{w} \rangle^2$. By putting the above together we get
            \[\frac{1}{5} \sum_{i=1}^{n} \langle c_i, \vec{w} \rangle ^4 + 3(\frac{1}{9} \sum_{i \neq j} \langle c_i, \vec{w} \rangle ^2 \langle c_j, \vec{w} \rangle ^2)\]
            and the final expression becomes 
            \[ mom_{4, \mat{C}} (\vec{w}) = \frac{1}{5} \sum_{i=1}^{n} \langle c_i, \vec{w} \rangle ^4 + \frac{1}{3} \sum_{i \neq j} \langle c_i, \vec{w} \rangle ^2 \langle c_j, \vec{w} \rangle ^2\]
    \end{enumerate}

    Now, since $\mat{C}$ is orthonormal, and by restricting $\vec{w}$ to the unit sphere in $\bb{R}^n$, we can simplify the expressions further.
    The second moment becomes $mom_{2, C}(\vec{w}) = \frac{1}{3} \vec{w} \mat{C}^t \mat{C} \vec{w}^t = \frac{1}{3}\vec{w} \mat{I} \vec{w}^t 
    = \frac{1}{3} \vec{w}\vec{w}^t = \frac{1}{3} \lVert \vec{w} \rVert ^2 = \frac{1}{3}$. \\
    By rewriting and expanding the fourth moment:
    \[ mom_{4, \mat{C}} (\vec{w}) = \frac{1}{5} \sum_{i=1}^{n} \langle c_i, \vec{w} \rangle ^4 + \frac{1}{3} \sum_{i \neq j} \langle c_i, \vec{w} \rangle ^2 \langle c_j, \vec{w} \rangle ^2\]
    \[ mom_{4, \mat{C}} (\vec{w}) = \frac{1}{5} \sum_{i=1}^{n} \langle c_i, \vec{w} \rangle ^4 + 
    \frac{1}{3} \sum_{i} \sum_{j} \langle c_i, \vec{w} \rangle ^2 \langle c_j, \vec{w} \rangle ^2 -
    \frac{1}{3}\sum_{i} \langle c_i, \vec{w} \rangle ^2 \langle c_i, \vec{w} \rangle ^2\]
    \[ mom_{4, \mat{C}} (\vec{w}) = \frac{1}{5} \sum_{i=1}^{n} \langle c_i, \vec{w} \rangle ^4 + 
    \frac{1}{3} \sum_{i} \sum_{j} \langle c_i, \vec{w} \rangle ^2 \langle c_j, \vec{w} \rangle ^2 -
    \frac{1}{3}\sum_{i} \langle c_i, \vec{w} \rangle ^4 \]
    \[ mom_{4, \mat{C}} (\vec{w}) = \frac{1}{3} \lVert \vec{w} \rVert ^4 - 
    \frac{2}{15}\sum_{i} \langle c_i, \vec{w} \rangle ^4 = \frac{1}{3} - 
    \frac{2}{15}\sum_{i} \langle c_i, \vec{w} \rangle ^4\]

    

\end{itemize}

\section{HPP against the Normal Distribution}

\begin{lemma}[Lemma 1]
\label{hpp_norm_lemma}
Assume signatures on the form $\vec{v} = \vec{x} \mat{V}$ where $x_i \sim \normdist{0}{\sigma}$. After converting $\PP{V}$ to $\PP{C}$, the fourth moment of $\PP{C}$ is constant over some $\vec{w}$ on the unit circle.
% \in \mathcal{O}](\bb{R})$
\end{lemma}
\begin{proof}
    Some proof here
\end{proof}
